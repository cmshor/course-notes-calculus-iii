\newlecture
\setcounter{chapter}{11}
\setcounter{section}{3}


%\def\textbookchapter{Chapter 11: Multiple Integrals}
\def\coursetopicnumber{III}
\def\topic{Applications of Double Integrals} % this is the printed title
\def\shorttopic{Applications of double integrals} % short topic
\def\textbookname{Active Calculus} % this is the corresponding textbook
\def\shorttextbookname{AC} % this is the short name for the book
\def\textbooksection{11.4} % corresponding textbook section
\def\textbooksectionurl{https://activecalculus.org/vector/S-11-4-Double-Integrals-Applications.html} % URL for textbook section
\def\handoutday{} % this is the printed date


%%%%%%%%% DOCUMENT CONTENT STARTS BELOW

\thispagestyle{plain}
\topstuff
\section{\topic{} \booklink{}}
\label{sec:double-integral-apps}
In single-variable calculus, we can use definite integrals to compute the average value of a function on an interval, the area of a region between two curves, and the mass of a one-dimensional object given a density function. We'll push these ideas to multivariable functions.

\subsection{Computing the size of the domain}
\subsubsection{Single-variable}
\begin{ex}
    For an interval $[a,b]$, draw and interpret $\displaystyle\int\limits_a^b \dx$.
\end{ex}

\vspace{1in}

\subsubsection{Multivariable}
\begin{ex}
    For a general region $R$, draw and interpret $\displaystyle\iint\limits_R \dA$.
\end{ex}

\vfill 

\begin{ex}
    Set up an iterated integral to compute the area of the region in the plane enclosed by circles of radius 3 and 4 centered at the origin.
\end{ex}

\vfill

\pagebreak 

\subsection{Average value of a function}
\subsubsection{Single-variable}
The \emph{average value} of a function $f(x)$ on an interval $I=[a,b]$ is
$f_{avg}=\phantom{\displaystyle\dfrac{1}{b-a}\int\limits_a^b f(x)\dx.}$

\vspace{1.5in}

\subsubsection{Multivariable}
The \emph{average value} of a function $f(x,y)$ on a region $R$ is 
$f_{avg}=\phantom{\displaystyle\dfrac{1}{\text{area}(R)}\iint\limits_R f\dA.}$

\vspace{1.5in}

\noindent Note: Since $\text{area}(R)=\displaystyle\iint\limits_R \dA$, we can rewrite the average value of $f$ on $R$ as 
\vspace{.5in}

\begin{ex}
    Set up an iterated integral to compute the average value of the function $f(x,y)=e^{-x^2-y^2}$ over the portion of the disc of radius 3 centered at the origin with $x\ge0$.
\end{ex}

\pagebreak 

\subsection{Domain splitting}
\subsubsection{Single-variable}
Suppose $a<b<c$ and that $f$ is continuous on $[a,c]$. Then $\displaystyle\int\limits_a^c f(x)\dx = \phantom{\int\limits_a^b f(x)\dx + \int\limits_b^c f(x)\dx.}$

\vspace{1in}

\subsubsection{Multivariable}
Suppose $R$ is the union of non-overlapping regions $R_1$ and $R_2$. Then for any function $f$ continuous on $R$, \\ \smallskip 

$\displaystyle\iint\limits_R f\dA = \phantom{\iint\limits_{R_1}f\dA + \iint\limits_{R_2}f\dA.}$

\vspace{1in}

\begin{ex}
    Consider the region $T$ consisting of all points in the triangle formed by the points $(0,0)$, $(6,2)$, and $(3,12)$.  For $f(x,y)=e^{xy}$, write $\displaystyle\iint\limits_T f\dA$ in terms of iterated integrals.
\end{ex}

\pagebreak 

\subsection{Volume of a region between surfaces}
\subsubsection{Single-variable}
If $g(x)\le h(x)$ for all $x$ values in an interval $[a,b]$, then the area of the region enclosed by the graphs of $y=g(x)$ and $y=h(x)$ over $[a,b]$ is \\ 
\smallskip 

\noindent Area = $\phantom{\displaystyle\int\limits_a^b \left(h(x)-g(x)\right)\dx.}$
\vspace{.5in}

\subsubsection{Multivariable}
If $g(x,y)\le h(x,y)$ for all points $(x,y)$ in a region $R$, then the volume of the region enclosed by the graphs of $z=g(x,y)$ and $z=h(x,y)$ over $R$ is \\
\smallskip 

\noindent Volume = $\phantom{\displaystyle\iint\limits_R\left(h(x,y)-g(x,y)\right)\dA}$
\vspace{1in}

Note: If the surfaces $z=g(x,y)$ and $z=h(x,y)$ enclose a region, we will typically need to find the intersection of the surfaces to determine the region $R$. We do this just as we did in the single variable case, equating the two functions and solving, thereby determining the boundary of $R$.
\begin{ex}
    Set up an iterated integral to compute the volume of the region enclosed by the graphs of $z=x^2+y^2$ and $z=2-x^2-y^2$.
\end{ex}

\vfill 

\pagebreak 

\subsection{Mass of an object}
\subsubsection{Single-variable}
Suppose we have an object that lies along the $x$-axis from $x=a$ to $x=b$. Suppose the density of the object is $\delta(x)$ (units of mass / units of $x$) at position $x$. Then the object has\\
\smallskip 

\noindent mass $= \displaystyle\int\limits_a^b \delta(x)\dx$
\vspace{.3in}

\subsubsection{Multivariable}
Suppose we have an object that fills a region $R$ in the $xy$-plane. Suppose the density of the object is $\delta(x,y)$ (units of mass / units of area) at position $(x,y)$. Then the object has \\
\smallskip 

\noindent mass $ = \displaystyle\iint\limits_R \delta(x,y)\dA$

\vspace{.3in}
Units are our guide! We can multiply together the units of the terms in the integral to get the units of the result.
\begin{ex}
    Suppose a flat disc of radius 3 cm has a density of $\frac{1}{1+r}$ grams per cm$^2$ for all points a distance of $r$ cm from the center. Set up an iterated integral to compute the mass of the disc.
\end{ex}
\vfill

\begin{ex}
    Same disc, but suppose the density at the point $(x,y)$ is $(3+x)$ g/cm$^2$. (The disc is centered at the origin.) Set up an iterated integral to compute the mass of the disc. Units?
\end{ex}
\vfill 
