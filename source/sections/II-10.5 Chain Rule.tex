\newlecture

\setcounter{section}{4}
%\def\textbookchapter{Chapter 10: Derivatives of Multivariable Functions}
\def\coursetopicnumber{II}
\def\topic{Chain Rule} % this is the printed title
\def\shorttopic{Chain rule} % short topic
\def\textbookname{Active Calculus} % this is the corresponding textbook
\def\shorttextbookname{AC} % this is the short name for the book
\def\textbooksection{10.5} % corresponding textbook section
\def\textbooksectionurl{https://activecalculus.org/vector/S-10-5-Chain-Rule.html} % URL for textbook section
\def\handoutday{} % this is the printed date

%%%%%%%%% DOCUMENT CONTENT STARTS BELOW

\thispagestyle{plain}
\topstuff
\section{\topic{} \booklink{}}
\label{sec:chain-rule}
\subsection{The chain rule in Calculus I}
Suppose $f$ is a function of $x$, and also suppose $x$ is a function of $t$. We have functions $f(x)$ and $x(t)$. We can compose them to get a new function of $t$: ``$f\circ x$,'' called ``$f$ composed with $x$,'' which is defined as 
\[
    (f\circ x) (t) = \phantom{f(x(t)).}
\] 
In Leibniz notation,
\[
    \dfd{f}{t}=\phantom{\dfd{f}{x}\dfd{x}{t}.}
\]
This gives us the function $f$ in terms of the variable $t$. 

\subsection{The chain rule in Calculus III, one independent variable}
Suppose $f$ is a function of variables $x$ and $y$, and suppose that $x$ and $y$ are each functions of a variable $t$. In other words, we have functions $f(x,y)$, $x(t)$, and $y(t)$.

From our work with differentials in Section~\ref{sec:linearization}, we know that 
\begin{align*}
    \dif f&=\phantom{f_x(x,y)\dx+f_y(x,y)\dy,}\\
    \dx&=\phantom{x'(t)\dt,}\\
    \dy&=\phantom{y'(t)\dt.}
\end{align*}

Subbing $\dx$ and $\dy$ into our expression for $\dif f$, we have
\[
    \dif f=\phantom{f_x(x,y)x'(t)\dt+f_y(x,y)y'(t)\dt.}
\]
This line, along with Leibniz notation, gives us a multivariable Chain Rule for the case of one independent variable ($t$ in this case).

\begin{thm}[Chain Rule with one independent variable]
    Let $f$ be a differentiable function of $x$ and $y$ on its domain (i.e., $f_x$ and $f_y$ exist), where $x$ and $y$ are differentiable functions of $t$ on some interval $I$ (i.e., $x'(t)$ and $y'(t)$ exist). Then 
    \[
        \dfd{f}{t}=\phantom{\pfp{f}{x}\cdot\dfd{x}{t}+\pfp{f}{y}\cdot\dfd{y}{t}.}
    \]
\end{thm}
\noindent \textbf{Note:} This theorem is written for a function $f$ of \emph{two} variables. If $f$ is a function of more than two variables, then the chain rule works exactly as you'd expect! For $f$ is a function of $x$, $y$, and $z$, and if $x$, $y$, and $z$ are functions of $t$, then 
\[
    \dfd{f}{t} = \hspace{4in}
\]

\pagebreak 

The product rule from Calculus I is a special case of the multivariable chain rule!
\begin{ex}
    For $f$ a function of $x$ and $y$, each of which is a function of $t$, suppose $f(x,y)=x\cdot y$. Compute $\displaystyle\dfd{f}{t}$.
\end{ex}

\vspace{1.5in}

\subsection{The chain rule in Calculus III, multiple independent variables}
Suppose $f$ is a function of variables $x$ and $y$, and suppose that $x$ and $y$ are each functions of variables $u$ and $v$. In other words, we have functions $f(x,y)$, $x(u,v)$ and $y(u,v)$.

From our work with differentials in Section~\ref{sec:linearization}, we know that 
\begin{align*}
    \dif f &= f_x(x,y)\dx+f_y(x,y)\dy,\\
    \dif x &= x_u(u,v)\dif u+x_v(u,v)\dif v,\\
    \dif y &= y_u(u,v)\dif u+y_v(u,v)\dif v.
\end{align*}
In other words, 
\[
    \dif f = \phantom{f_x(x,y)\big[x_u(u,v)\dif u+x_v(u,v)\dif v\big]+f_y(x,y)\big[y_u(u,v)\dif u+y_v(u,v)\dif v\big].}
\]
This line, with Leibniz and partial derivative notation, gives us a multivariable chain rule for the case of two independent variables ($u$ and $v$ in this case). Note that $\pfp{u}{v}=0$ and $\pfp{v}{u}=0$.

\begin{thm}[Chain Rule with two independent variables]
    Let $f$ be a differentiable function of $x$ and $y$, where $x$ and $y$ are differentiable functions of $u$ and $v$. Then
    \[
        \pfp{f}{u}= \hspace{2.6in} \text{ and } \hspace{.3in}
        \pfp{f}{v}=\hspace{2.6in}
    \]
\end{thm}

%We'll see how to deal with other scenario using \emph{tree diagrams}.

\begin{ex}
    Let $f(x,y)=3xy-5y^2+4$, \quad $x=2u+v$, \quad and $y=3u-v$. Compute $\displaystyle\pfp{f}{u}$.
\end{ex}

\pagebreak 

\iffalse
\begin{ex}\label{ex:chain-rule-one-var}
Let $f(x)=x^2+3x+4$ and let $x(t)=3t+1$. \\

Compute $(f\circ x)(t)$. Then compute its derivative.
\end{ex}
\vfill

The \emph{chain rule} allows us to calculate its derivative with respect to $t$:
\[
    \hspace{-1in}\dd{t}\Big(f(x(t))\Big) = \phantom{f'(x(t))\cdot x'(t).}
\]
In Leibniz notation, this is written as 
\[
    \hspace{-1in}\dfd{f}{t} = \phantom{\dfd{f}{x} \cdot \dfd{x}{t}.}
\]

This scales up if we have a longer chain of variables. If $f(x)$, $x(t)$, and $t(u)$ are functions, then we can think of $f$ as a function of $u$: $f(x(t(u)))$. Then 
\[
    \hspace{-1in}\dfd{f}{u} = \phantom{\dfd{f}{x}\cdot\dfd{x}{t}\cdot\dfd{t}{u}.}
\]
\begin{ex}
Let $f(x)=x^2+3x+4$ and let $x(t)=3t+1$.
\begin{enumerate}
\item Compute $\dfd{f}{x}$ and $\dfd{x}{t}$. 
\item Multiply them together and compare the result to the derivative in Exercise \ref{ex:chain-rule-one-var}.\end{enumerate}
\end{ex}
\vfill

\pagebreak 

\section{The chain rule in Calculus III}
\subsection{Version 1 -- One independent variable}
Say we have a function $f$ which depends on $x$ and $y$. It could be that $x$ and $y$ both depend on some variable $t$. As $t$ varies, $x$ and $y$ vary, so $f$ varies. Thus, $f$ is a function of $t$, so we can talk about its derivative. I.e., we can talk about $\dfd{f}{t}$.
\begin{ex}
Let $f(x,y)=3xy-5y^2+4$, \quad $x=t^2$, \quad and $y=3t$. \\

Write $f$ as a function of $t$ and compute $\dfd{f}{t}$.
\end{ex}
\vfill

\begin{thm}[Chain Rule with one independent variable]
Let $f$ be a differentiable function of $x$ and $y$ on its domain (i.e., $f_x$ and $f_y$ exist), where $x$ and $y$ are differentiable functions of $t$ on some interval $I$ (i.e., $x'(t)$ and $y'(t)$ exist). Then 
\[
    \dfd{f}{t}=\phantom{\pfp{f}{x}\cdot\dfd{x}{t}+\pfp{f}{y}\cdot\dfd{y}{t}.}
\]
\end{thm}
\noindent \textbf{Note:} This theorem is written for a function $f$ of \emph{two} variables. If $f$ is a function of more than two variables, then the chain rule works exactly as you'd expect! For $f$ is a function of $x$, $y$, and $z$, and if $x$, $y$, and $z$ are functions of $t$, then 
\[\dfd{f}{t} = \hspace{4in}\]
%\vspace{.5in}

\begin{ex}
Let $f(x,y)=3xy-5y^2+4$, \quad $x=t^2$, \quad and $y=3t$. \\

Compute $\pfp{f}{x}$, $\pfp{f}{y}$, $\dfd{x}{t}$, $\dfd{y}{t}$ and use the theorem above to compute $\dfd{f}{t}$.
\end{ex}
\vfill
\pagebreak 
\subsection{Version 2 -- More than one independent variable}
Say we have a function $f$ which depends on $x$ and $y$. It could be that $x$ and $y$ both depend on some variables $u$ and $v$. Thus, $f$ is a function of $u$ and $v$, so we can talk about its partial derivatives with respect to $u$ and $v$. I.e., we can talk about $\pfp{f}{u}$ and $\pfp{f}{v}$.
\begin{ex}
Let $f(x,y)=3xy-5y^2+4$, \quad $x=2u+v$, \quad and $y=3u-v$.\\

Write $f$ as a function of $u$ and $v$, and then compute $\pfp{f}{u}$ and $\pfp{f}{v}$.
\end{ex}
\vfill

\begin{thm}[Chain Rule with two independent variables]
Let $f$ be a differentiable function of $x$ and $y$, where $x$ and $y$ are differentiable functions of $u$ and $v$. Then 
\[\hspace{-1in}\pfp{f}{u}=\phantom{\pfp{f}{x}\cdot\pfp{x}{u}+\pfp{f}{y}\cdot\pfp{y}{u}} \quad \text{ and } \quad \pfp{f}{v}=\phantom{\pfp{f}{x}\cdot\pfp{x}{v} + \pfp{f}{y}\cdot\pfp{y}{v}.}\]
\end{thm}
\noindent \textbf{Note}: We have a similar result when there are more variables. Tree diagrams (next page) will help!
\begin{ex}
Let $f(x,y)=3xy-5y^2+4$, \quad $x=2u+v$, \quad and $y=3u-v$.\\

Compute $\pfp{f}{x}$, $\pfp{f}{y}$, $\pfp{x}{u}$, $\pfp{x}{v}$, $\pfp{y}{u}$, $\pfp{y}{v}$.  Then use the theorem above to compute $\pfp{f}{u}$ and $\pfp{f}{v}$.
\end{ex}
\vfill
\pagebreak
\fi 

\subsection{Tree diagrams}
We can use a \emph{tree diagram} to assist in remembering the chain rule, no matter how many variables we have.

If we have some function (say $a$) that is a function of some variables (say $b$ and $c$), then we write $a$ on one level and write $b$ and $c$ on the level below, connecting $a$ to both $b$ and $c$ with a straight line. We'll mark those lines with the corresponding derivative or partial derivative (in this case $\pfp{a}{b}$ and $\pfp{a}{c}$). Repeat this process with every variable that depends on another variable or variables.

Now, if we have a variable (say $f$) that is in some level above another variable (say $t$), then we can compute the derivative or partial derivative of $f$ with respect to $t$:
\begin{itemize}
    \item For each path from $f$ down to $t$, multiply the derivatives or partial derivatives together. 
    \item Add up the results for every path from $f$ down to $t$.
    \item The result is $\dfd{f}{t}$ (if $t$ is the only variable in its level) or $\pfp{f}{t}$ (if there are other variables in the level of $t$).
\end{itemize}

\begin{ex}
    Say $f$ is a function of $x$, $y$, and $z$. Say each of $x$, $y$, and $z$ is a function of $u$ and $v$.\\
    
    Draw the tree diagram and determine $\pfp{f}{v}$.
\end{ex}

\vfill

\begin{ex}
    Say $A$ is a function of $m$, $m$ is a function of $z$ and $r$, and $z$ and $r$ are functions of $s$ and $j$. Draw the tree diagram and determine $\pfp{A}{j}$.
\end{ex}

\vspace{2in}

\iffalse
\pagebreak 

\section{Implicit differentiation revisited}
Recall that if $x$ and $y$ are related by some implicit equation (like $x^2+y^2=1$), then we can compute $\dfd{y}{x}$ using implicit differentiation. The chain rule gives us a streamlined approach for this calculation.

\begin{ex}
    Suppose we have the equation $F(x,y)=0$ for some function $F$, where $y$ is implicitly defined as a function of $x$. Apply the chain rule to both sides of the equation and solve for $\dfd{y}{x}$.
\end{ex}\vspace{1in}

\begin{thm}
    Let $F$ be differentiable and suppose that $F(x,y)=0$ defines $y$ implicitly as a function of $x$. Then, assuming $F_y\ne0$,
    \[\dfd{y}{x} = \phantom{-\dfrac{F_x}{F_y}}\]
\end{thm}\vspace{1in}

\begin{ex}
    Use the above result to find $\dfd{y}{x}$ given that $x^2+y^2=1$.
\end{ex}
\fi 
