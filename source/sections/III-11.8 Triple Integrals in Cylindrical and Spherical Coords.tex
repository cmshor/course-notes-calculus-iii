\newlecture
\setcounter{chapter}{11}
\setcounter{section}{7}

%\def\textbookchapter{Chapter 11: Multiple Integrals}
\def\coursetopicnumber{III}
\def\topic{Triple Integrals in Cylindrical and Spherical Coordinates} % this is the printed title
\def\shorttopic{Cylindrical, spherical integration} % short topic
\def\textbookname{Active Calculus} % this is the corresponding textbook
\def\shorttextbookname{AC} % this is the short name for the book
\def\textbooksection{11.8} % corresponding textbook section
\def\textbooksectionurl{https://activecalculus.org/vector/S-11-8-Triple-Integrals-Cylindrical-Spherical.html} % URL for textbook section
\def\handoutday{} % this is the printed date

%%%%%%%%% DOCUMENT CONTENT STARTS BELOW

\thispagestyle{plain}
\topstuff
\section{\topic{} \booklink{}}
\label{sec:triple-spherical-cylindrical}
In Section \ref{sec:double-int-polar}, we saw how to evaluate double integrals using polar coordinates. This choice of coordinates is useful when integrating over a region that has circular symmetry.

When working with triple integrals, we will see two new coordinate systems. Cylindrical coordinates are useful when a region has circular symmetry. Spherical coordinates are useful when a region has spherical symmetry.

\subsection{Cylindrical coordinates}
Cylindrical coordinates are useful for describing a region in $\mathbb{R}^3$ that has some sort of circular symmetry. To describe a point $P$ in $\mathbb{R}^3$, we visualize it as if it is on the rim of a circular cylinder with its base centered at the origin in the $xy$-plane. We describe $P$ with a radius $r$, a height $z$, and an angle $\theta$ on the rim of the face of the cylinder.
\bigskip 

\noindent A point $P=(x,y,z)$ (rectangular coordinates) can be written with \emph{cylindrical coordinates} $(r,\theta,z)$: % as:
\begin{itemize}
    \item $x=\phantom{r\cos(\theta)}$
    \item $y=\phantom{r\sin(\theta)}$
    \item $z=\phantom{z}$
\end{itemize}

\vfill

%We generally take $r\ge0$,\, $0\le \theta<2\pi$,\, and $z\in\mathbb{R}$.
%\subsection{Fundamental surfaces}
%\vspace{2in}


\begin{ex}
    Use cylindrical coordinates to describe a wedge of a cheese wheel of radius 6 cm that is 4 cm tall and has an angle of $30^\circ$.
\end{ex}

\vspace{1.5in}

\pagebreak 

\subsubsection{Differentials for cylindrical coordinates}
To compute  $\iiint\limits_D f\dV$ with cylindrical coordinates, we need $\dV$ in terms of $\dr$, $\dtheta$, $\dz$. \medskip 

\noindent Since cylindrical coordinates are just polar coordinates with $z$, and since $\dV=\dx\dy\dz$, we have \bigskip 

\[
    \dV=\phantom{ \dx\dy\dz=r\dr\dtheta\dz.}\hspace{2in}
\] 

\bigskip 

\begin{framed}
    {\centering 
        \textbf{Method to compute a triple integral using cylindrical coordinates}  
    \par}
    \bigskip 
    
    Given a function $f$ and a region $D$, to compute $\iiint\limits_D f\dV$ using cylindrical coordinates,
    \begin{itemize}
        \item Describe $D$ using inequalities with cylindrical coordinates $r$, $\theta$, $z$.
        \item Put those bounds in as the endpoints of each integral symbol, recalling the order rules.
        \item Write $f$ in terms of cylindrical coordinates (using $x=r\cos\theta$, $y=r\sin\theta$, $z=z$). Keep in mind that $x^2+y^2=r^2$.
        \item Write $\dV = r\dr\dtheta\dz$, with the order of differentials corresponding to the order of the integrals.
        \item Evaluate Inner, then Middle, then Outer. If any integral is difficult, think about putting the differentials in another valid order (if possible).
    \end{itemize} 
\end{framed}

\begin{ex}
    Let $f(x,y,z)=x-2y+z$. Let $D$ be the solid cylinder of radius 3 and height 4 sitting on the $xy$-plane centered on the positive $z$-axis. Write $\displaystyle\iiint\limits_D f\dV$ using cylindrical coordinates.
\end{ex}

\vfill

\pagebreak 
\subsection{Spherical coordinates}
Spherical coordinates are useful for describing a region in $\mathbb{R}^3$ that has some sort of spherical symmetry. To describe a point $P$ in $\mathbb{R}^3$, we think about it as being on a sphere of radius $\rho$ along with two angles: $\theta$ and $\phi$, which are like longitude and latitude.

\subsection{Coordinates}
A point $P=(x,y,z)$ (rectangular coordinates) can be written with \emph{spherical coordinates} $(\rho,\theta,\phi)$:
\begin{itemize}
    \item $x=\phantom{\rho\cos(\theta)\sin(\phi)}$
    \item $y=\phantom{\rho\sin(\theta)\sin(\phi)}$
    \item $z=\phantom{\rho\cos(\phi)}$
\end{itemize}

\vfill 

%We generally take $\rho\ge0$,\, $0\le \theta<2\pi$, and $0 \le \phi \le \pi$.
%\vfill 

%\subsection{Fundamental surfaces}
%\vspace{2in}

\begin{ex}
    Use spherical coordinates to describe the portion of a ball of radius 5 centered at the origin with $y\le0$ and $z\ge0$. %(A \emph{ball} is a filled-in sphere, just as a \emph{disk} is a filled-in circle.)
\end{ex}

\vspace{1.5in}

\pagebreak 

\subsubsection{Differentials for spherical coordinates}
To compute $\displaystyle\iiint\limits_D f\dV$ with spherical coordinates, we need $\dV$ in terms of $\drho$, $\dtheta$, $\dphi$.

If you're at the point $(\rho,\theta,\phi)$ and increase each variable (by $\Delta \rho$, $\Delta \theta$, $\Delta \phi$), the resulting volume is 
\[
    \Delta V \approx \rho^2 \sin\phi \, \Delta \rho\, \Delta \theta \, \Delta \phi.
\]
To understand where this comes from, see \href{https://activecalculus.org/vector/S-11-8-Triple-Integrals-Cylindrical-Spherical.html#A_11_8_8}{Activity 11.8.6 in the textbook}. Thus, for integration purposes, 
\[
    \dV = \phantom{\rho^2\sin\phi \, \drho \dtheta \dphi.}
\]
\bigskip 

(More generally, to understand how differentials work with a change of coordinates for any number of variables, see \href{https://activecalculus.org/vector/S-11-9-Change-of-Variable.html}{Section 11.9 of the textbook, which is titled ``Change of Variables.''})
\begin{framed}
    {\centering 
        \textbf{Method to compute a triple integral using spherical coordinates}  
    \par}
    \bigskip 
    
    Given a function $f$ and a region $D$, to compute $\iiint\limits_D f\dV$ using spherical coordinates,
    \begin{itemize}
        \item Describe $D$ using inequalities with spherical coordinates $\rho$, $\theta$, $\phi$.
        \item Put those bounds in as the endpoints of each integral symbol, recalling the order rules.
        \item Write $f$ in terms of spherical coordinates (using $x=\rho\cos\theta\sin\phi$, $y=\rho\sin\theta\sin\phi$, $z=\rho\cos\phi$). Keep in mind that $x^2+y^2+z^2=\rho^2$.
        \item Write $\dV = \rho^2\sin\phi\drho\dtheta\dphi$, with the order of differentials corresponding to the order of the integrals.
        \item Evaluate Inner, then Middle, then Outer. If any integral is difficult, think about putting the differentials in another valid order (if possible).
    \end{itemize} 
\end{framed}

\begin{ex}
    Let $f(x,y,z)=3+x^2+y^2+z^2$, and let $W$ be the portion of a ball of radius 5 centered at the origin where $y\le0$ and $z\ge0$. Write $\displaystyle\iiint\limits_W f\dV$ as using spherical coordinates.
\end{ex}

\pagebreak 

\subsection{Examples}
Remember, the choice of coordinates depends on the region!
\begin{ex}
    A circular cone $D$ is given by the equation $z=\sqrt{x^2+y^2}$. Suppose a circular cone of height 10 cm is filled with ice cream. The density of the ice cream $z$ cm above the bottom is $(30-z)$g/cm$^3$. Set up a triple integral to compute the mass of the ice cream in the cone using cylindrical coordinates.
\end{ex}

\vfill 

\pagebreak 

\begin{ex}
    Let $D$ be the top half of a ball of radius 7 centered at the origin. Compute the volume of $D$ using a triple integral.
    %Using geometry, determine $\displaystyle\iiint_D\dV$. Write this triple integral as an iterated integral using Cartesian coordinates. Write this triple integral as an iterated integral using spherical coordinates.%Rewrite $\displaystyle\int\limits_{0}^7\int\limits_{-\sqrt{49-z^2}}^{\sqrt{49-z^2}}\int\limits_{-\sqrt{49-y^2-z^2}}^{\sqrt{49-y^2-z^2}}\dx\dy\dz$ as an integral with spherical coordinates. Then evaluate it.%Compute the volume of a sphere of radius 7.
\end{ex}

\vfill

\vspace{.5in}

\pagebreak 

\begin{ex}
    Set up an iterated integral to find the volume of the solid region $D$ which is enclosed by the top half of a unit sphere centered at the origin and a circular cone given by the equation $z=\sqrt{x^2+y^2}$. (If time permits, evaluate it.)
\end{ex}

