\newlecture
\setcounter{chapter}{11}
\setcounter{section}{4}

%\def\textbookchapter{Chapter 11: Multiple Integrals}
\def\coursetopicnumber{III}
\def\topic{Double Integrals in Polar Coordinates} % this is the printed title
\def\shorttopic{Double integrals, polar coordinates} % short topic
\def\textbookname{Active Calculus} % this is the corresponding textbook
\def\shorttextbookname{AC} % this is the short name for the book
\def\textbooksection{11.5} % corresponding textbook section
\def\textbooksectionurl{https://activecalculus.org/vector/S-11-5-Double-Integrals-Polar.html} % URL for textbook section
\def\handoutday{} % this is the printed date

%%%%%%%%% DOCUMENT CONTENT STARTS BELOW

\thispagestyle{plain}
\topstuff
(We are covering Section 11.5, which is on double integrals in polar coordinates, before we cover Section 11.4, which is on applications of double integrals.)
\section{\topic{} \booklink{}}
\label{sec:double-int-polar}
\subsection{Polar coordinates}
Our usual way of representing a point $P$ in the $xy$-plane is with \emph{Cartesian} (or \emph{rectangular}) coordinates $x$ and $y$. For instance, if $P=(5,2)$, then the origin and $P$ are opposite corners on a $5\times2$ rectangle with edges along the $x$- and $y$-axes.
\vspace{1.5in}

We can use circles in place of rectangles. This leads to \emph{polar} coordinates $r$ and $\theta$. Any point $P$ lies on a circle of radius $r\ge0$ centered at the origin at an angle $\theta$ above the positive $x$-axis.

\vfill

\begin{ex}
    Convert $(r,\theta)=(7,\pi/3)$ to rectangular coordinates $(x,y)$. Then convert $(x,y)=(-4,0)$ to polar coordinates $(r,\theta)$.
\end{ex}

\vspace{1in}

\pagebreak 

\subsection{Describing regions with polar coordinates}
\begin{ex}
    Draw the region $R$ of points $P$ that have $1\le r\le 2$ and $\pi/4\le\theta\le \pi$.
\end{ex}

\vspace{1.5in}

\begin{ex}
    Use polar coordinates to describe the region between circles of radii 3 and 5 centered at the origin.
\end{ex}

\vspace{2in}

%\subsection{Double integration over a polar region}
\subsection{Area of a subrectangle in rectangular coordinates}
Suppose you're at the point $P=(x,y)$. Draw a rectangle of width $\Delta x$ and height $\Delta y$ at $P$. The area of the rectangle is 
\[
    \Delta A = \hspace{7in} 
\]

\vfill 
Thus, with rectangular coordinates, the double integral of a function $f$ over a region $R$ is \bigskip 

\[
    \iint\limits_R f \dA = \hspace{3in} 
\]

\vspace{1in}

\pagebreak 

\subsection{Area of a ``subrectangle'' in polar coordinates}
Suppose you're at the point $P=(r,\theta)$. Draw a polar ``rectangle'' with angle $\Delta \theta$ and radius $\Delta r$. Its area is \[\Delta A = \hspace{7in}\]
\vspace{2in}

Thus, with polar coordinates, the double integral of a function $f$ over a region $R$ is \bigskip 

\[
    \iint \limits_R f \dA = \hspace{3in} 
\]

\medskip 

\subsection{Evaluating double integrals using polar coordinates}
\begin{framed}
    \noindent 
    To evaluate a double integral of a function $f$ over a region $R$, the choice of coordinates (rectangular or polar) depends entirely on the region $R$. It does not depend on the function $f$.
\end{framed}
\bigskip 

Essentially, if $R$ can be easily described using polar coordinates, which typically occurs when $R$ contains all of or a portion of a disc, we will do the following: 
\begin{itemize} 
    \item describe $R$ with inequalities for $r$ and $\theta$; 
    \item write $f$ using polar coordinates (using $x=r\cos\theta$, $y=r\sin\theta$); 
    \item write $\dA=r \dr \dtheta$ (or $\dA=r \dtheta \dr$); and 
    \item take the iterated integral approach, using the bounds from the inequalities for $r$ and $\theta$ as the iterated integral endpoints, matching the order of the inner and outer integrals with the order of $\dr$ and $\dtheta$.
\end{itemize} 
\pagebreak 

\begin{ex}
    Let $R$ be the top left quarter of a disc of radius 2 centered at the origin. Describe $R$ with rectangular coordinates and with polar coordinates. Then, for $f(x,y)=3x+1$, evaluate $\iint\limits_R f\dA$ using either rectangular or polar coordinates. 
\end{ex}
\vfill

\begin{ex}
    Find the volume of the solid bounded by the paraboloid $z=9-x^2-y^2$ and the $xy$-plane.
\end{ex}
\vfill 

\pagebreak

\begin{ex}
    Evaluate the following iterated integral by first converting it to polar coordinates:
    \medskip 
    
    \noindent $\displaystyle \int\limits_0^{\sqrt{2}/2} \int\limits_x^{\sqrt{1-x^2}} 4xy\dy\dx$
\end{ex}
\vfill 

\begin{ex}
    Evaluate $\displaystyle\int\limits_0^6 \int\limits_{-\sqrt{36-x^2}}^{\sqrt{36-x^2}} \ee^{-3x^2-3y^2}\dy\dx$.
\end{ex}
\vfill 

%\begin{ex}
%    Find the volume of the region beneath the surface $z=xy+10$ and above the annular region $R$ described with polar coordinate inequalities by $R=\{(r,\theta) : 2\le r\le 4,\, 0\le \theta\le 2\pi\}$.
%\end{ex}
