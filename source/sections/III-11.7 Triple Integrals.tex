\newlecture
\setcounter{chapter}{11}
\setcounter{section}{6}

%\def\textbookchapter{Chapter 11: Multiple Integrals}
\def\coursetopicnumber{III}
\def\topic{Triple Integrals} % this is the printed title
\def\shorttopic{Triple integrals} % short topic
\def\textbookname{Active Calculus} % this is the corresponding textbook
\def\shorttextbookname{AC} % this is the short name for the book
\def\textbooksection{11.7} % corresponding textbook section
\def\textbooksectionurl{https://activecalculus.org/vector/S-11-7-Triple-Integrals.html} % URL for textbook section
\def\handoutday{} % this is the printed date

%%%%%%%%% DOCUMENT CONTENT STARTS BELOW

\thispagestyle{plain}
\topstuff
\section{\topic{} \booklink{}}
\label{sec:triple-integrals}
\subsection{Single, double, triple}
\begin{itemize}
    \item For $f$ a function of one variable and $I$ an interval in the number line, a \emph{single integral} is an integral of the form 
    \[
        \phantom{\int\limits_I f\dx.}
    \]
    $\dx$ is a differential that represents an infinitesimally small unit of length.
    %\vfill 

    \item For $f$ a function of two variables and $R$ a region in the $xy$-plane, a \emph{double integral} is an integral of the form 
    \[
        \phantom{\iint\limits_R f\dA.}
    \]
    $\dA$ is a differential that represents an infinitesimally small unit of area.
    %\vfill\vfill
    
    \item For $f$ a function of three variables and $D$ a region in 3-dimensional space, a \emph{triple integral} is an integral of the form 
    \[
        \phantom{\iiint\limits_D f\dV.}
    \]
    $\dV$ is a differential that represents an infinitesimally small unit of volume.
    %\vfill\vfill
\end{itemize}

In Section \ref{sec:double-integral-apps}, we saw some double integral applications. They also apply to triple integrals.

\begin{itemize}
    \item 
    If $f$ is a function of 3 variables and $D$ is a region in 3-space, then the (signed) volume (?!) of the 4D region below (?!) the graph of $w=f(x,y,z)$ above the region $D$ is
    %\vfill %
    \[
        \phantom{\iiint\limits_D f\dV.}
    \]
    
    \item If an object fills a region $D$ in 3-space, and the density of the object at the point $(x,y,z)$ is $\delta(x,y,z)$ (units of mass per unit of volume), then the total mass of the object is %\vfill
    \[
        \phantom{\iiint\limits_D \delta \dV.}
    \]
    
    \item If $D$ is a region in 3-space, then its volume is 
    %\vfill %
    $\phantom{\displaystyle\iiint \limits_D \dV.}$
    
    \item If $D$ is a region in 3-space, then the average value of a function $f$ on $D$ is 
    %\vfill %
    $\phantom{\displaystyle\dfrac{1}{\text{vol}(D)}\iiint\limits_D f\dV.}$
\end{itemize}

\pagebreak 



\pagebreak 

\subsection{Triple integral, general 3D region, Cartesian coordinates}
Let $D$ be a region in 3-dimensions, and suppose we can describe $D$ by 
\[ 
    a\le x\le b, \quad\quad
    g(x)\le y\le h(x), \quad\quad
    p(x,y) \le z \le q(x,y) 
\]
for constants $a, b$, and functions $g(x)$, $h(x)$, $p(x,y)$, $q(x,y)$.

Then for any function $f$ of $x$, $y$, $z$, we have $\dV=\dz\dy\dx$, so 
\[
    \iiint\limits_D f\dV = 
    \phantom{\int\limits_a^b \int\limits_{g(x)}^{h(x)} \int\limits_{p(x,y)}^{q(x,y)} f(x,y,z)\dz\dy\dx.}
\]

If, instead, we had inequalities with a different order of variables, such as
\[ 
    c\le y\le d, \quad\quad
    g(y)\le z\le h(y), \quad\quad
    p(y,z) \le x \le q(y,z), 
\]
then we would have 
\[
    \iiint\limits_D f\dV = 
    \phantom{\int\limits_c^d \int\limits_{g(y)}^{h(y)} \int\limits_{p(y,z)}^{q(y,z)} f(x,y,z)\dx\dz\dy.}
\]
Of course, there are other possibilities for the inequalities. The differentials must match up with the integral symbols. (Inner, Middle, Outer.) 
\begin{itemize} 
    \item In general, the bounds for Inner can involve variables from %Middle and/or Outer. 
    \\ 
    \item The bounds for Middle can involve the variable from %Outer. 
    \\ 
    \item The bounds for Outer are %constants.
    \\
\end{itemize} 

\noindent 
There are generally two main stages in computing a triple integral:
\begin{enumerate}
    \item \mbox{} \\ 
    \item \mbox{} \\ 
\end{enumerate}

\subsection{Describing 3D regions}
It can be difficult to describe a 3D region! However, we have some tricks. The main idea is to determine bounds for one of the variables (say $z$) in terms of the others (say $x$ and $y$). Then we can project the 3D region into a region in the $xy$-plane. Then we need to describe the region in the $xy$-plane, which we can using either rectangular coordinates (Section \ref{sec:double-integral-rectangle}) or polar coordinates (Section \ref{sec:double-int-polar}).

\pagebreak 

\subsection{Examples}
\begin{ex}
    Set up and evaluate $\displaystyle\iiint\limits_D 6x\dV$ for $D$ the region given by
    \[0\le y\le 1,\quad y\le z\le x,\quad y\le x\le 2y.\]
\end{ex}
\vfill 

\begin{ex}
    Let $W$ be the top half of the unit ball centered at the origin. Describe $W$ with inequalities for $x$, $y$, and $z$. (A \emph{ball} is a filled-in sphere, just as a \emph{disc} is a filled-in circle.) Then, for any function $f$, write down $\iiint\limits_W f\dV$ as an iterated integral.
\end{ex}

\vfill\vspace{1in}

\pagebreak 
\begin{ex}
    How would you determine the average height ($z$-value) of a point in the top half of a ball of radius 3?
\end{ex}

\vspace{1.5in}

\begin{ex}
    A building $B$ has a rectangular base that is 8m wide and 16m long. It has a flat roof that is slanted so that one corner is 12m high and its adjacent corners are 10m high.
    \begin{enumerate}
        \item Describe the building with inequalities for $x$, $y$, $z$.
        \item Set up an iterated integral to compute the building volume.% (which is $\displaystyle\iiint\limits_B \dV$).
        \item Suppose the building is filled with a gas that has density $\delta(x,y,z)=(xy+z)$ kg/m$^3$. Set up an iterated integral to compute the total mass of the gas in the building.
    \end{enumerate}
\end{ex}

\vfill

%\pagebreak 

\pagebreak 

\begin{ex}
    Consider the region $D$ in the first octant under the plane $x+2y+3z=6$. For some function $f(x,y,z)$, write $\displaystyle\iiint\limits_D f\,\dV$ as an iterated integral with differentials in order $\dy \dx \dz$. %Then write it as an iteraged integral with differentials in order $\dy \dx \dz$.
\end{ex}
\vfill

