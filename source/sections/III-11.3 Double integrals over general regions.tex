\newlecture
\setcounter{chapter}{11}
\setcounter{section}{2}

%\def\textbookchapter{Chapter 11: Multiple Integrals}
\def\coursetopicnumber{III}
\def\topic{Double Integrals over General Regions} % this is the printed title
\def\shorttopic{Double integrals, general regions} % short topic
\def\textbookname{Active Calculus} % this is the corresponding textbook
\def\shorttextbookname{AC} % this is the short name for the book
\def\textbooksection{11.3} % corresponding textbook section
\def\textbooksectionurl{https://activecalculus.org/vector/S-11-3-Double-Integrals-General.html} % URL for textbook section
\def\handoutday{} % this is the printed date

%%%%%%%%% DOCUMENT CONTENT STARTS BELOW

\thispagestyle{plain}
\topstuff
\section{\topic{} \booklink{}}
\label{sec:double-int-general-region}

\subsection{Double integrals over rectangles}
The simplest kind of region $R$ in the $xy$-plane is a rectangle. Any rectangle can be described as the set of points $(x,y)$ where 
\[
    a\le x\le b \quad \text{ and } \quad c\le y\le d
\] 
for some constants $a,b,c,d$ with $a\le b$ and $c\le d$. 

In order to compute the double integral of some function $f(x,y)$ over such a rectangle $R$, we use \emph{iterated integrals} as described in Section~\ref{sec:double-integral-rectangle}:
\[
    \iint\limits_R f(x,y)\dA 
    = \int\limits_{y=c}^{y=d} \left(\,\int\limits_{x=a}^{x=b} f(x,y)\dx\right) \dy 
    = \int\limits_{x=a}^{x=b}\left(\,\int\limits_{y=c}^{y=d} f(x,y)\dy\right)\dx.
\] 
This gives us the (signed) volume under the graph of $z=f(x,y)$ over the region $R$.

Often we will get lazy and write this without parentheses and/or without equal signs in the endpoints of the integral:
\[
    \iint\limits_R f(x,y)\dA 
    = \int\limits_{c}^{d} \int\limits_{a}^{b} f(x,y)\dx \dy 
    = \int\limits_{a}^{b} \int\limits_{c}^{d} f(x,y)\dy \dx .
\] 
\textbf{Note that the order of the differentials matters. It lets us know which variables our endpoints represent.}

\subsection{Descriptions of general regions}
Before we can think about double integrals over general regions, we need to see how to describe general regions with inequalities.

\begin{center}
    \textbf{The key to doing this well is sketching graphs and labeling functions \\ and intersection points.}
\end{center} 
This process is similar to those of computing areas of regions between curves and computing volumes of solids of revolution in Calculus II.

Our typical approach to describe a region $R$ is to describe it as the set of points $(x,y)$ that satisfy inequalities in one of the following ways:
\[
    a\le x\le b\quad \text{ and } \quad g(x)\le y\le h(x)\hspace{2in}\mbox{}
\] 
\[
    \text{ or }\hspace{2in}\mbox{}
\] 
\[
    c\le y\le d \quad \text{ and } \quad g(y)\le x\le h(y)\hspace{2in}\mbox{}
\]
\pagebreak 

\begin{ex}
    Draw the region $R$ described by $-1\le x\le 2$ and $x^2\le y\le 5$.
\end{ex}

\vfill

\begin{ex}
    Draw the region $R$ described by $0\le y\le 6$ and $-y/2\le x\le y/2$.
\end{ex} 

\vfill 

\pagebreak 

\begin{ex}\label{ex:region-inequalities}
    In the $xy$-plane, consider the triangle $T$ that has vertices $(2,0)$, $(6,0)$, and $(6,8)$. Represent this triangle with inequalities in both ways.
\end{ex}

\vfill

\begin{ex}
    Using inequalities, describe the region $R$ in the $xy$-plane bounded by the graphs of $y=4x$ and $y=x^2$.  %Do this both ways (using bottom and top functions; and using left and right functions).
\end{ex}

\vfill



\pagebreak 
\begin{ex}
    Draw the region $R$ in the $xy$-plane bounded by the graphs of $y=2x$, $y=6-x$, and the $y$-axis. Label each curve and intersection points. Then describe it with inequalities.
\end{ex}

\vfill 

\begin{ex}
    Draw the region $R$ in the $xy$-plane bounded by the graphs of $y=2x$, $y=6-x$, and $y=1$. Label each curve and intersection points. Then describe it with inequalities.
\end{ex}

\vfill 

\pagebreak 

\subsection{Integrating over general regions via iterated integrals}
Once we have described a region $R$ with inequalities, we can compute a double integral of a function $f(x,y)$ over $R$ via iterated integrals. This allows us to compute the (signed) volume of 3D region below the surface $z=f(x,y)$ and above the region $R$.
\begin{itemize}
    \item If $R$ is described by $a\le x\le b$ and $g(x)\le y\le h(x)$, then \[\iint\limits_R f(x,y)\dA = \int\limits_{x=a}^{x=b} \left(\,\int\limits_{y=g(x)}^{y=h(x)}f(x,y)\dy\right)\dx.\]
    More simply, 
    \[
        \iint\limits_R f(x,y)\dA = \hspace{2in}
    \]
    %\int\limits_a^b\int\limits_{g(x)}^{h(x)}f(x,y)\dy \dx.\]
    \item If $R$ is described by $c\le y\le d$ and $g(y)\le x\le h(y)$, then 
    \[
        \iint\limits_R f(x,y)\dA = \int\limits_{y=c}^{y=d} \left(\,\int\limits_{x=g(y)}^{x=h(y)}f(x,y)\dx\right)\dy.
    \]
    More simply, 
    \[
        \iint\limits_R f(x,y)\dA = \hspace{2in}
    \]
    %\int\limits_c^d\int\limits_{g(y)}^{h(y)}f(x,y)\dx \dy.\]
\end{itemize}
We then proceed as we did before with iterated integrals, evaluating the inner integral first, and then evaluating that in the outer integral.

\begin{example}
    In Exercise~\ref{ex:region-inequalities}, we described the triangle $T$ with the inequalities in two ways.
    Our first description of $T$ was
    \begin{center}
        $2\le x\le 6$ \quad and \quad  $0\le y\le 2x-4$.
    \end{center}
    Thus, for any function $f(x,y)$, the double integral of $f$ over $T$ is
    \[\iint\limits_T f\dA = \phantom{\int\limits_2^6 \int\limits_0^{2x-4} f(x,y)\dy\dx.}\]
    Our second description of $T$ was
    \begin{center}
        $0\le y\le 8$ \quad and \quad $(y+4)/2\le x\le 6$.
    \end{center}
    Thus, for any function $f(x,y)$, the double integral of $f$ over $T$ is 
    \[\iint\limits_T f\dA = \phantom{\int\limits_0^8 \int\limits_{(y+4)/2}^6 f(x,y)\dx\dy.}\]
    
    This provides flexibility, as the inner integral may be easier to integrate in terms of one variable than it is in terms of the other.
\end{example}
\pagebreak 
\begin{ex}
    Evaluate the iterated integral $\displaystyle\int\limits_{-2}^0\int\limits_{0}^{\sqrt{4-x^2}} y\dy \dx$. Also, draw the region $R$ that this double integral is over.
\end{ex}
\vfill 

\begin{ex}
    Evaluate the following double integral (perhaps changing the limits of integration).
    \[
        \int\limits_0^2\int\limits_y^2 \ee^{x^2}\dx\dy
    \]
\end{ex}


\vfill
